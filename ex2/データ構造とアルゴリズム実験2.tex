\documentclass[10.5pt,a4paper]{jsarticle}
\usepackage{graphicx}
\usepackage{amsmath}

\usepackage[final]{listings}
\usepackage{color}

\normalsize

\lstset{
  language=fortran,
  basicstyle=\ttfamily\scriptsize,
  commentstyle={\ttfamily \color[cmyk]{0,1,1,0}},
  keywordstyle={\bfseries \color[cmyk]{1,1,0,0}},
  stringstyle={\ttfamily \color[cmyk]{1,0,1,0.5}},
  stepnumber=1,
  numberstyle=\ttfamily,
  breaklines=true,
  breakindent=20pt,
  frame=tblr,
  framesep=4pt,
  tabsize=2
}

\newcommand{\figref}[1]{図\ref{#1}}
\newcommand{\tabref}[1]{表\ref{#1}}
\newcommand{\secref}[1]{\ref{#1}節}

\title{データ構造とアルゴリズム実験レポート\\課題2:\textless Javaによるプログラミングの復讐\textgreater}
\author{\textless 201811320\textgreater \textless 1クラス\textgreater \textless 江頭輝\textgreater}
\date{締切日:\textless 2019年4月22日\textgreater\\提出日:\textless 2019年4月27日\textgreater}

\begin{document}
\maketitle

\section{必須課題2-1}

この課題では,セルに整数値を格納する連結リストを実現するクラスListをJava言語で実装する。

\subsection{List.javaの作成}
\subsubsection{実装の方針}

まず,\texttt{List}クラスを定義する。
セルの直後(ポインタは)`p.next`のような形で表現し、セルの値は`p.data`のように表記する。


\subsubsection{実装コードおよびコードの説明}

\begin{figure}[t]
  \begin{center}
   \lstinputlisting[language=java,numbers=left]{List.java}
   \caption{List.javaのソースコード}
  \end{center}
\end{figure}

\subsubsection{実装の方針}
insert\_cellとinsert\_cell\_topでは即席変数であるnew\_cellを作成しそのデータに引数を渡すことで新たなセルを作成する。
topの場合はのポインタにこれまでのtexttt{head}をよび、
前者では引数として渡したのポインタがのポインタになるようにする。
delete\_cellではのポインタに\texttt{p}のポインタのポインタを代入することでセルの削除を行うようにしている。
delete\_cell\_topではではなくtexttt{head}になった。
displayではのポインタが\texttt{null}になるまで値をループで出力する。


\subsubsection{実行結果}\label{sec:ls_code_exec}

まず,List.javaを以下のコマンドでコンパイルする.
\begin{verbatim}
-------------------------------------
$ javac List.java
-------------------------------------
\end{verbatim}
コンパイル後に以下のコマンドで実行を行った。
\begin{verbatim}
-------------------------------------
$ java List
123
13
3
-------------------------------------
\end{verbatim}

\texttt{main}メゾッドでまず、1を先頭に追加、先頭の直後に3, 2を順番に追加していった。
なので、43行目の\texttt{display}関数で出力っされる値は\texttt{123}である。
次に\texttt{head}セルの次のセルを削除したので46行目の\texttt{display}関数では13が出力される。
最後に\texttt{delete\_cell\_top}関数を呼び出し、先頭のセルを削除したため、49行目での\texttt{display}関数では3のみが出力された。

\subsubsection{考察}

このコードだと、headの先頭にセルを追加する場合新しいリストを作る必要があります。
また、headのみ仕様が異なるので、余分に\texttt{insert\_cell\_top}メゾッドと\texttt{delete\_cell\_top}メゾッドを呼ぶ必要がありました。
それを踏まえると、ダミーのデータ部を持つheadを作成するほうが実装が単純になると考えられます。

\section{必須課題2-2}

\subsection{QueueArray.javaの作成}

\begin{figure}[t]
  \begin{center}
   \lstinputlisting[language=java,numbers=left]{QueueArray.java}
   \caption{QueueArray.javaのソースコード}
   \label{code:QueueArray}
  \end{center}
 \end{figure}

\subsubsection{実装の方針}\label{sec:bs_impl_policy}
\texttt{QueueArray}クラスにはエンキューを行うための\texttt{enqueue}メゾッドとデキューを行うための\texttt{dequeue}。
キュー全体を表示する\texttt{display}メゾッドを作成した。そしてその動作を確認するための\texttt{main}メゾッドを作成する。

1. QueueArrayクラスのコンストラクタ

キューの作成に必要なものとしてコンストラクタで
整数の配列を表す\texttt{queue}、
格納できるメモリ量を表す長さを表す\texttt{length}、
キューの先頭を\texttt{front}
末尾を\texttt{rear}と定義した。

2. enqueueメゾッド

まずオーバーフローの処理を行う。もともとの配列Queueを超えるような値の挿入が起こってしまう可能性があるからだ。\\
オーバーフローしないことがわかったら要素の末尾に引数で指定した値を代入してrearに適切な処理を行う。

3. dequeueメゾッド

まずアンダーフローの処理を行う。それはキューの要素が存在しない場合にデキューを行わないようにするためである。\\
アンダーフローの処理後に要素の先頭の処理をしてその値を返すようにする。

4. displayメゾッド

要素の位置はコンストラクタで指定した\texttt{front}と\texttt{rear}をつかって表現することができるので、
その間の値を返すように実装を行う。

\subsubsection{実装コードおよびコードの説明}

\figref{code:QueueArray}QueueArray.javaのソースコードを示す.
\secref{sec:bs_impl_policy}で述べた,
\texttt{enqueue}メソッドと
\texttt{dequeue}メソッドは,それぞれ
14$\sim$21行目,
24$\sim$32行の部分に相当する.

\texttt{enqueue}
は末尾が配列の長さより大きくなる場合はオーバーフローとなり終了する。15$\sim$18行目でその実装をしている。
オーバーフローを起こす場合、\texttt{System.exit(1)}を書いて終了することができる。
オーバーフローを起こさない場合、\texttt{queue[rear]}が要素の末尾にあたるので
そこに引数として与えたint型の変数\texttt{val}を代入する。
キューのサイズを変えるためにrearの値を一つ上げる。そうすることで要素の末尾は右に一つずれたような挙動になる.

\texttt{dequeue}
は要素の中身がない場合はアンダーフローとなり終了する。
25$\sim$28行目でその実装をしている。frontとrearが等しいときに\texttt{dequeue}を行うと、アンダーフローになるので、その条件のときに\texttt{Queue Underflow!!}と出力して
先程と同様に\texttt{System.exit(1)}によりプログラムを終了させる。
デキューされるのは\texttt{queue[front]}であるのでその値を返す値である\texttt{x}に代入する。
引数の値を代入した後にfrontを一つ動かし要素を拡大する。

\subsubsection{実行結果}

\secref{sec:ls_code_exec}で述べたコンパイル方法と同様に,
QueueArray.javaを\texttt{javac}コマンドでコンパイルし,
QueueArray.classを生成する.その後,以下のコマンドを入力することによって,
QueueArray.javaのプログラムを実行する.

\begin{verbatim}
-------------------------------------
$ java QueueArray
12
1
2
-------------------------------------
\end{verbatim}

配列の長さが10のQUEUEを作成し、そこに順番に1, 2をエンキューする。
displayすると\texttt{12}が出力される。

デキューをすると1次に2が出力される。
この状態でデキューを行うとunderflow!が出力されて終了することも確認できた。

\subsubsection{考察}
\texttt{enqueue}メゾッドの計算量は\texttt{O(1)}であり、
\texttt{dequeue}メゾッドの計算量も\texttt{O(1)}である。
これはあらかじめ先頭と末尾の位置を変数で格納していることにより実現できた。
\end{document}

\section{発展課題2-3}

\subsection{双方向循環リストの作成}

\begin{figure}[t]
  \begin{center}
   \lstinputlisting[language=java,numbers=left]{No3/ListDL.java}
   \caption{ListDL.javaのソースコード}
   \label{code:ListDL}
  \end{center}
 \end{figure}

\subsubsection{実装の方針}\label{sec:bs_impl_policy}
\texttt{ListDL}クラスにはエンキューを行うための\texttt{enqueue}メゾッドとデキューを行うための\texttt{dequeue}。
キュー全体を表示する\texttt{display}メゾッドを作成した。そしてその動作を確認するための\texttt{main}メゾッドを作成する。
キューの作成に必要なものとしてコンストラクタで整数の配列を表す\texttt{queue}、格納できるメモリ量を表す長さを表す\texttt{length}、キューの先頭を\texttt{front}末尾を\texttt{rear}と定義した。

\subsubsection{実装コードおよびコードの説明}

\figref{code:ListDL}ListDL.javaのソースコードを示す.
\secref{sec:bs_impl_policy}で述べた,
\texttt{enqueue}メソッドと
\texttt{dequeue}メソッドは,それぞれ
14$\sim$21行目,
24$\sim$32行の部分に相当する.
\texttt{enqueue}は末尾が配列の長さより大きくなる場合はオーバーフローとなり終了する。rearの値を一つ上げるとキューの先頭は右に一つずれる。
\texttt{dequeue}はキューの中身がない場合はアンダーフローとなり終了する。引数の値を代入する際にfrontを一つ動かしキューを拡大する。

\subsubsection{実行結果}

\secref{sec:ls_code_exec}で述べたコンパイル方法と同様に,
ListDL.javaを\texttt{javac}コマンドでコンパイルし,
ListDL.classを生成する.その後,以下のコマンドを入力することによって,
ListDL.javaのプログラムを実行する.

\begin{verbatim}
-------------------------------------
$ java ListDL
12
1
2
-------------------------------------
\end{verbatim}
\subsubsection{考察}
\texttt{enqueue}メゾッドの計算量は\texttt{O(1)}であり、
\texttt{dequeue}メゾッドの計算量も\texttt{O(1)}である。
これはあらかじめ先頭と末尾の位置を変数で格納していることにより実現できた。
\end{document}

\section{必須課題2-4}

\subsection{双方向循環リストの改良}

\begin{figure}[t]
  \begin{center}
   \lstinputlisting[language=java,numbers=left]{ListDL.java}
   \caption{ListDL.javaのソースコード}
   \label{code:ListDL}
  \end{center}
 \end{figure}

\subsubsection{実装の方針}\label{sec:bs_impl_policy}
\texttt{ListDL}クラスにはエンキューを行うための\texttt{enqueue}メゾッドとデキューを行うための\texttt{dequeue}。
キュー全体を表示する\texttt{display}メゾッドを作成した。そしてその動作を確認するための\texttt{main}メゾッドを作成する。
キューの作成に必要なものとしてコンストラクタで整数の配列を表す\texttt{queue}、格納できるメモリ量を表す長さを表す\texttt{length}、キューの先頭を\texttt{front}末尾を\texttt{rear}と定義した。

\subsubsection{実装コードおよびコードの説明}

\figref{code:ListDL}ListDL.javaのソースコードを示す.
\secref{sec:bs_impl_policy}で述べた,
\texttt{enqueue}メソッドと
\texttt{dequeue}メソッドは,それぞれ
14$\sim$21行目,
24$\sim$32行の部分に相当する.
\texttt{enqueue}は末尾が配列の長さより大きくなる場合はオーバーフローとなり終了する。rearの値を一つ上げるとキューの先頭は右に一つずれる。
\texttt{dequeue}はキューの中身がない場合はアンダーフローとなり終了する。引数の値を代入する際にfrontを一つ動かしキューを拡大する。

\subsubsection{実行結果}

\secref{sec:ls_code_exec}で述べたコンパイル方法と同様に,
ListDL.javaを\texttt{javac}コマンドでコンパイルし,
ListDL.classを生成する.その後,以下のコマンドを入力することによって,
ListDL.javaのプログラムを実行する.

\begin{verbatim}
-------------------------------------
$ java ListDLmain
12
1
2
-------------------------------------
\end{verbatim}
\subsubsection{考察}
\texttt{enqueue}メゾッドの計算量は\texttt{O(1)}であり、
\texttt{dequeue}メゾッドの計算量も\texttt{O(1)}である。
これはあらかじめ先頭と末尾の位置を変数で格納していることにより実現できた。
\end{document}